\thispagestyle{plain}
\begin{center}       
    \large
    \vspace{0.9cm}
    \textbf{Data Mining for the LoRaWAN}
        
    \vspace{0.4cm}
    by \textbf{Ante Lojić Kapetanović}
    
\end{center}
The state of the art Internet of Things (IoT) networks are characterized with the long range connectivity, high robustness, energy efficiency and scalability as key performance properties.
The Long Range Wide Area Network (LoRaWAN) is the media access control (MAC) protocol for wide area networks designed to perfectly fit the mentioned needs. 
The LoRaWAN allows long range communication at a very low data rate thus reducing the power consumption, which is the primary goal of a modern IoT deployment.
This thesis explores the predicability of the IoT traffic which is assumed to be based on independent and unpredictable activations of devices.
The information about the probability of the end device activation for near future would result in tremendous progress: more efficient access protocols could be designed, which would subsequently result in a better throughput, reliability, latency, and also energy efficiency.
Firstly, most of the theoretical aspects of the LoRaWAN are covered in the introductory part of the thesis.
The structuring of the measured data in a time series format and the proposal of the predictive method follows.
The Long Short Term Memory (LSTM) neural network model is proposed as a way to predict the end device activations.
Finally, based on the predictive model results, the estimation of the probability of the end device activations is shown.
The discussion of the results and the application of the prediction model, along with potential improvements of the model itself, are presented in the conclusion.\\

\textbf{Keywords:} IoT, LoRaWAN, time series, LSTM
