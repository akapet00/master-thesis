\thispagestyle{plain}
\begin{center}       
    \large
    \vspace{0.9cm}
    \textbf{Rudarenje podataka LoRaWAN mreže}
        
    \vspace{0.4cm}
    autor: \textbf{Ante Lojić Kapetanović}
    
\end{center}
Posljednja dostignuća i ključna svojstva Internet stvari (IoT) mreža su dalekosežno povezivanje uređaja, robustnost, energetska učinkovitost i skalabilnost mreže.
Dalekosežna mreža širokog područja (LoRaWAN) je definirana kao protokol kontrole pristupa mediju (MAC) za mreže širokog područja i dizajnirana je tako da zadovoljava navedena ključna svojstva.
LoRaWAN mreža osigurava dalekosežnu komunikaciju pri niskoj brzini prijenosa podataka, smanjujući tako potrošnju energije što je osnovni cilj modernih implementacija IoT mreža.
Ovaj rad se bavi mogućnošću predviđanja ponašanja IoT mrežnog prometa, koji se temelji na komunikaciji neovisnih i nepredvidljivih krajnjih uređaja.
Predikcijom aktivnosti krajnjih uređaja za blisku budućnost, osigurala bi se mogućnost definiranja učinkovitijih pristupnih protokola, što bi posljedično rezultiralo boljom mrežnom propusnošću i pouzdanošću same mreže ali i energetskom efikasnošću.
Kroz uvodni dio rada opisani su teoretski aspekti LoRaWAN mreže. 
Nakon toga je definirana struktura snimanog mrežnog prometa, pretvorba u vremensku seriju i prijedlog prediktivnog modela.
Prediktivni model za predviđanje aktivnosti krajnjih uređaja je sekvencijalna LSTM neuralna mreža.
Konačno, temeljem izlaznih podataka korištene neuralne mreže, procjena vjerojatnosti aktivnosti krajnjih uređaja promatrane mreže je prikazana u posljednjem dijelu rada.
Diskusija i interpretacija rezultata, kao i moguća primjena razvijenog prediktivnog modela te njegove mane su prezentirane u zaključku.\\

\textbf{Ključne riječi:} IoT, LoRaWAN, vremenske serije, LSTM
