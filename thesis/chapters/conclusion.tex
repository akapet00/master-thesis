% SAŽETI PRIKAZ MOTIVACIJE ZA RAD U ŠIREM PODRUČJU TEME 
Through this thesis, the prediction model for the end device activation in the Long Range Wide Area Network (LoRaWAN) was proposed.
The LoRaWAN has recently emerged as an open sourced media access control protocol built on top of proprietary physical layer, LoRa, employing the CSS modulation technique.
The LoRaWAN is used for low power wide area networks (LPWAN) where the power efficient communication over very long distances is required.
These kind of networks are deployed for wireless sensor networks, where the activation of an end device is triggered with an event.
Without the scheduling on the base station of a network, collisions occur incessantly.  
This causes retransmissions and, consequently, a significantly greater energy consumption.
With the possibility of the end device activation predicition, communication protocols could be upgraded, but since there is a low chance of the prediction working in the stohastic environment, alternative forecasting methods had to be introduced.

% TEMA RADA, GLAVNI ELEMENTI ONOGA ŠTO JE NAPRAVLJENO
%1. TEMA RADA (1 REČENICA)
%2. SADRŽAJ UVODNOG DIJELA (2-4 REČENICE)
%3. SADRŽAJ PRAKTIČNOG DIJELA (2-4 REČENICE)
The Long Short-Term Memory (LSTM) neural network prediction model was developed and trained on the time series data set from the real LoRaWAN deplyment. 
An overview of the LoRaWAN technology and forecasting methods for a time series data were investigated during the theoretical part of the thesis. 
The LSTM model was created using Python programming language and Keras neural-network library, which runs on top of TensorFlow GPU. 

% SAŽETI PRIKAZ OSTVARENIH REZULTATA, VLASTITI ZAKLJUČCI TEMELJEM OSTVARENIH REZ
% OPCI ZAKLJUCCI O PRIMJENJENOM I REALIZIRANOM SUSTAVU I METODI
The predictive, sequntial model (implementation in \autoref{chap:appendix}) had been trained on the large time series data set where only the activation of the observed device for specific moment was captured.
Output values for each input sequence of historical data varied from 0 to 1.
Closer the value is to 0, there is a smaller chance that the actual activation for the future moment will happen.
Root mean square error (RMSE), considering time distance between the actual and the predicted event, was less than 2s.
The real problem is multiplied predicted activation for some moments, where instead of single activation, multiple activation is predicted.
In fig. \ref{fig:prob}, it is shown that the probability of an end device activation for the observed future time period raises as the time period raises.

Even though the results look promising, there are a few issues.
Firstly, only data from a single device was observed. 
In a wireless sensor network there are hundreds and more of devices, which means that creating data and training multiple models for each device would be extremely slow and exhausting.
Moreover, this model is adjusted for univariate time series data and as such, correlations with other devices are not taken into account. 
Lastly, becuase of the hardware limitation, the model was trained using 10-second sequences of historical data. 
Longer training sequences would not only improve the model itself, but also output could be sequnce predictions insted of point-to-point predictions.